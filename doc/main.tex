% generated by GAPDoc2LaTeX from XML source (Frank Luebeck)
\documentclass[a4paper,11pt]{report}

\usepackage{a4wide}
\sloppy
\pagestyle{myheadings}
\usepackage{amssymb}
\usepackage[latin1]{inputenc}
\usepackage{makeidx}
\makeindex
\usepackage{color}
\definecolor{FireBrick}{rgb}{0.5812,0.0074,0.0083}
\definecolor{RoyalBlue}{rgb}{0.0236,0.0894,0.6179}
\definecolor{RoyalGreen}{rgb}{0.0236,0.6179,0.0894}
\definecolor{RoyalRed}{rgb}{0.6179,0.0236,0.0894}
\definecolor{LightBlue}{rgb}{0.8544,0.9511,1.0000}
\definecolor{Black}{rgb}{0.0,0.0,0.0}

\definecolor{linkColor}{rgb}{0.0,0.0,0.554}
\definecolor{citeColor}{rgb}{0.0,0.0,0.554}
\definecolor{fileColor}{rgb}{0.0,0.0,0.554}
\definecolor{urlColor}{rgb}{0.0,0.0,0.554}
\definecolor{promptColor}{rgb}{0.0,0.0,0.589}
\definecolor{brkpromptColor}{rgb}{0.589,0.0,0.0}
\definecolor{gapinputColor}{rgb}{0.589,0.0,0.0}
\definecolor{gapoutputColor}{rgb}{0.0,0.0,0.0}

%%  for a long time these were red and blue by default,
%%  now black, but keep variables to overwrite
\definecolor{FuncColor}{rgb}{0.0,0.0,0.0}
%% strange name because of pdflatex bug:
\definecolor{Chapter }{rgb}{0.0,0.0,0.0}
\definecolor{DarkOlive}{rgb}{0.1047,0.2412,0.0064}


\usepackage{fancyvrb}

\usepackage{mathptmx,helvet}
\usepackage[T1]{fontenc}
\usepackage{textcomp}


\usepackage[
            pdftex=true,
            bookmarks=true,        
            a4paper=true,
            pdftitle={Written with GAPDoc},
            pdfcreator={LaTeX with hyperref package / GAPDoc},
            colorlinks=true,
            backref=page,
            breaklinks=true,
            linkcolor=linkColor,
            citecolor=citeColor,
            filecolor=fileColor,
            urlcolor=urlColor,
            pdfpagemode={UseNone}, 
           ]{hyperref}

\newcommand{\maintitlesize}{\fontsize{50}{55}\selectfont}

% write page numbers to a .pnr log file for online help
\newwrite\pagenrlog
\immediate\openout\pagenrlog =\jobname.pnr
\immediate\write\pagenrlog{PAGENRS := [}
\newcommand{\logpage}[1]{\protect\write\pagenrlog{#1, \thepage,}}
%% were never documented, give conflicts with some additional packages

\newcommand{\GAP}{\textsf{GAP}}

%% nicer description environments, allows long labels
\usepackage{enumitem}
\setdescription{style=nextline}

%% depth of toc
\setcounter{tocdepth}{1}





%% command for ColorPrompt style examples
\newcommand{\gapprompt}[1]{\color{promptColor}{\bfseries #1}}
\newcommand{\gapbrkprompt}[1]{\color{brkpromptColor}{\bfseries #1}}
\newcommand{\gapinput}[1]{\color{gapinputColor}{#1}}


\begin{document}

\logpage{[ 0, 0, 0 ]}
\begin{titlepage}
\mbox{}\vfill

\begin{center}{\maintitlesize \textbf{\textsf{kreher-stinson}\mbox{}}}\\
\vfill

\hypersetup{pdftitle=\textsf{kreher-stinson}}
\markright{\scriptsize \mbox{}\hfill \textsf{kreher-stinson} \hfill\mbox{}}
{\Huge \textbf{Algorithms from the book implemented in GAP\mbox{}}}\\
\vfill

{\Huge Version 0.1\mbox{}}\\[1cm]
{8 April 2016\mbox{}}\\[1cm]
\mbox{}\\[2cm]
{\Large \textbf{Bert{\a'\i}n Hern{\a'a}ndez-Trejo  \mbox{}}}\\
{\Large \textbf{ Rafael Villarroel-Flores   \mbox{}}}\\
{\Large \textbf{Citlalli Zamora-Mej{\a'\i}a  \mbox{}}}\\
\hypersetup{pdfauthor=Bert{\a'\i}n Hern{\a'a}ndez-Trejo  ;  Rafael Villarroel-Flores   ; Citlalli Zamora-Mej{\a'\i}a  }
\end{center}\vfill

\mbox{}\\
{\mbox{}\\
\small \noindent \textbf{Bert{\a'\i}n Hern{\a'a}ndez-Trejo  }  Email: \href{mailto://bertin13@gmail.com} {\texttt{bertin13@gmail.com}}}\\
{\mbox{}\\
\small \noindent \textbf{ Rafael Villarroel-Flores   }  Email: \href{mailto://rvf0068@gmail.com} {\texttt{rvf0068@gmail.com}}\\
  Homepage: \href{http://rvf0068.github.io} {\texttt{http://rvf0068.github.io}}}\\
{\mbox{}\\
\small \noindent \textbf{Citlalli Zamora-Mej{\a'\i}a  }  Email: \href{mailto://cizame@gmail.com} {\texttt{cizame@gmail.com}}}\\
\end{titlepage}

\newpage\setcounter{page}{2}
{\small 
\section*{Copyright}
\logpage{[ 0, 0, 1 ]}
 {\copyright} 2016 by Bert{\a'\i}n Hern{\a'a}ndez-Trejo, Rafael
Villarroel-Flores and Citlalli Zamora-Mej{\a'\i}a

 \textsf{kreher-stinson} package is free software; you can redistribute it and/or modify it under the
terms of the \href{http://www.fsf.org/licenses/gpl.html} {GNU General Public License} as published by the Free Software Foundation; either version 2 of the License,
or (at your option) any later version. \mbox{}}\\[1cm]
\newpage

\def\contentsname{Contents\logpage{[ 0, 0, 2 ]}}

\tableofcontents
\newpage

 
\chapter{\textcolor{Chapter }{Generating Combinatorial Objects}}\label{generating}
\logpage{[ 1, 0, 0 ]}
\hyperdef{L}{X817B7D9F78E96B39}{}
{
  
\section{\textcolor{Chapter }{Subsets}}\label{subsets}
\logpage{[ 1, 1, 0 ]}
\hyperdef{L}{X87BB51CD81E9D34A}{}
{
  

\subsection{\textcolor{Chapter }{KSSubsetLexRank}}
\logpage{[ 1, 1, 1 ]}\nobreak
\hyperdef{L}{X7DF749F9873EBFAC}{}
{\noindent\textcolor{FuncColor}{$\triangleright$\ \ \texttt{KSSubsetLexRank({\mdseries\slshape n, T})\index{KSSubsetLexRank@\texttt{KSSubsetLexRank}}
\label{KSSubsetLexRank}
}\hfill{\scriptsize (function)}}\\


 Returns the rank of \mbox{\texttt{\mdseries\slshape T}} as a subset of the set of numbers from 1 to \mbox{\texttt{\mdseries\slshape n}} (Algorithm 2.1). An error is produced if \mbox{\texttt{\mdseries\slshape T}} is not a subset of the set $\{1..n\}$. 
\begin{Verbatim}[commandchars=!@|,fontsize=\small,frame=single,label=Example]
  !gapprompt@gap>| !gapinput@KSSubsetLexRank(4,[1,2,3]);|
  14
  !gapprompt@gap>| !gapinput@KSSubsetLexRank(4,[2,4]);|
  5
  !gapprompt@gap>| !gapinput@KSSubsetLexRank(4,[]);|
  0
  !gapprompt@gap>| !gapinput@KSSubsetLexRank(4,[1,2,3,4]);|
  15
  !gapprompt@gap>| !gapinput@KSSubsetLexRank(4,[1,2,3,4,5]);|
  Error, the set [ 1, 2, 3, 4, 5 ] is not a subset of [1 ..4]
\end{Verbatim}
 }

 

\subsection{\textcolor{Chapter }{KSSubsetLexUnrank}}
\logpage{[ 1, 1, 2 ]}\nobreak
\hyperdef{L}{X85B1636183D7350D}{}
{\noindent\textcolor{FuncColor}{$\triangleright$\ \ \texttt{KSSubsetLexUnrank({\mdseries\slshape n, r})\index{KSSubsetLexUnrank@\texttt{KSSubsetLexUnrank}}
\label{KSSubsetLexUnrank}
}\hfill{\scriptsize (function)}}\\


 Returns the subset of $\{1..\mbox{\texttt{\mdseries\slshape n}}\}$ whose rank is \mbox{\texttt{\mdseries\slshape r}}. (Algorithm 2.2). The number $r$ has to be greater than $0$ and less than $2^n-1$. 
\begin{Verbatim}[commandchars=!@|,fontsize=\small,frame=single,label=Example]
  !gapprompt@gap>| !gapinput@KSSubsetLexUnrank(4,14);|
  [ 1, 2, 3 ]
  !gapprompt@gap>| !gapinput@KSSubsetLexUnrank(4,5);|
  [ 2, 4 ]
  !gapprompt@gap>| !gapinput@KSSubsetLexUnrank(4,0);|
  [  ]
  !gapprompt@gap>| !gapinput@KSSubsetLexUnrank(4,15);|
  [ 1, 2, 3, 4 ]
  !gapprompt@gap>| !gapinput@KSSubsetLexUnrank(4,17);|
  Error, there is no subset of [1 ..4] of rank 17
\end{Verbatim}
 }

 

\subsection{\textcolor{Chapter }{KSkSubsetLexRank}}
\logpage{[ 1, 1, 3 ]}\nobreak
\hyperdef{L}{X83526EE47D6343F1}{}
{\noindent\textcolor{FuncColor}{$\triangleright$\ \ \texttt{KSkSubsetLexRank({\mdseries\slshape T, k, n})\index{KSkSubsetLexRank@\texttt{KSkSubsetLexRank}}
\label{KSkSubsetLexRank}
}\hfill{\scriptsize (function)}}\\


 Finds the rank of \mbox{\texttt{\mdseries\slshape T}}, among all \mbox{\texttt{\mdseries\slshape k}}-subsets of $\{1,2,\ldots,n\}$. If \mbox{\texttt{\mdseries\slshape T}} is not a \mbox{\texttt{\mdseries\slshape k}}-subset of $\{1,2,\ldots,n\}$, then an error is produced. 
\begin{Verbatim}[commandchars=!@|,fontsize=\small,frame=single,label=Example]
  !gapprompt@gap>| !gapinput@KSkSubsetLexRank([1,2,3],3,5);|
  0
  !gapprompt@gap>| !gapinput@KSkSubsetLexRank([1,3,4],3,5);|
  3
  !gapprompt@gap>| !gapinput@KSkSubsetLexRank([3,4,5],3,5);|
  9
  !gapprompt@gap>| !gapinput@KSkSubsetLexRank([1,2,3,4],3,5);|
  Error, the set [ 1, 2, 3, 4 ] is not a 3-subset of [1 .. 5]
  !gapprompt@gap>| !gapinput@KSkSubsetLexRank([1,3,6],3,5);|
  Error, the set [ 1, 3, 6 ] is not a 3-subset of [1 .. 5]
\end{Verbatim}
 }

 

\subsection{\textcolor{Chapter }{KSkSubsetLexUnrank}}
\logpage{[ 1, 1, 4 ]}\nobreak
\hyperdef{L}{X807C3B077BCCA874}{}
{\noindent\textcolor{FuncColor}{$\triangleright$\ \ \texttt{KSkSubsetLexUnrank({\mdseries\slshape r, k, n})\index{KSkSubsetLexUnrank@\texttt{KSkSubsetLexUnrank}}
\label{KSkSubsetLexUnrank}
}\hfill{\scriptsize (function)}}\\


 Given an integer \mbox{\texttt{\mdseries\slshape r}} between 0 and ${n \choose k}-1$, returns the \mbox{\texttt{\mdseries\slshape k}}-subset of an \mbox{\texttt{\mdseries\slshape n}}-set with rank \mbox{\texttt{\mdseries\slshape r}}. 
\begin{Verbatim}[commandchars=!@|,fontsize=\small,frame=single,label=Example]
  !gapprompt@gap>| !gapinput@KSkSubsetLexUnrank(0,3,5);|
  [ 1, 2, 3 ]
  !gapprompt@gap>| !gapinput@KSkSubsetLexUnrank(3,3,5);|
  [ 1, 3, 4 ]
  !gapprompt@gap>| !gapinput@KSkSubsetLexUnrank(9,3,5);|
  [ 3, 4, 5 ]
  !gapprompt@gap>| !gapinput@KSkSubsetLexUnrank(-1,3,5);|
  Error, there is no 3-subset of [1 .. 5] of rank -1
  !gapprompt@gap>| !gapinput@KSkSubsetLexUnrank(10,3,5);|
  Error, there is no 3-subset of [1 .. 5] of rank 10
\end{Verbatim}
 }

 }

 
\section{\textcolor{Chapter }{Permutations}}\label{Permutations}
\logpage{[ 1, 2, 0 ]}
\hyperdef{L}{X80F808307A2D5AB8}{}
{
  

\subsection{\textcolor{Chapter }{KSPermLexRank}}
\logpage{[ 1, 2, 1 ]}\nobreak
\hyperdef{L}{X81A50E8C7ED8344D}{}
{\noindent\textcolor{FuncColor}{$\triangleright$\ \ \texttt{KSPermLexRank({\mdseries\slshape n, pi})\index{KSPermLexRank@\texttt{KSPermLexRank}}
\label{KSPermLexRank}
}\hfill{\scriptsize (function)}}\\


 Given a permutation \mbox{\texttt{\mdseries\slshape pi}} of $\{1..\mbox{\texttt{\mdseries\slshape n}}\}$, returns the rank of \mbox{\texttt{\mdseries\slshape pi}}. (Algorithm 2.15) }

 

\subsection{\textcolor{Chapter }{KSPermLexUnrank}}
\logpage{[ 1, 2, 2 ]}\nobreak
\hyperdef{L}{X79A6234B7CDC00F1}{}
{\noindent\textcolor{FuncColor}{$\triangleright$\ \ \texttt{KSPermLexUnrank({\mdseries\slshape n, r})\index{KSPermLexUnrank@\texttt{KSPermLexUnrank}}
\label{KSPermLexUnrank}
}\hfill{\scriptsize (function)}}\\


 Returns the permutation of $\{1..\mbox{\texttt{\mdseries\slshape n}}\}$ with rank \mbox{\texttt{\mdseries\slshape r}}. (Algorithm 2.16) }

 }

 }

 
\chapter{\textcolor{Chapter }{Bactracking}}\label{backtracking}
\logpage{[ 2, 0, 0 ]}
\hyperdef{L}{X82385A297FD5AFF5}{}
{
  
\section{\textcolor{Chapter }{Knapsack}}\label{knapsack}
\logpage{[ 2, 1, 0 ]}
\hyperdef{L}{X872264E47F34A84F}{}
{
  

\subsection{\textcolor{Chapter }{KSCheckKnapsackInput}}
\logpage{[ 2, 1, 1 ]}\nobreak
\hyperdef{L}{X7FB05C427FBC528B}{}
{\noindent\textcolor{FuncColor}{$\triangleright$\ \ \texttt{KSCheckKnapsackInput({\mdseries\slshape K})\index{KSCheckKnapsackInput@\texttt{KSCheckKnapsackInput}}
\label{KSCheckKnapsackInput}
}\hfill{\scriptsize (function)}}\\


 Checks for valid input data for the Knapsack problems (Problems 1.1-1.4). \mbox{\texttt{\mdseries\slshape K}} is a list, which first element is the vector of profits, the second is the
vector of weights, and the third is the capacity of the knapsack, which must
be an integer. }

 

\subsection{\textcolor{Chapter }{KSKnapsack1}}
\logpage{[ 2, 1, 2 ]}\nobreak
\hyperdef{L}{X84E39E7E8371D3BB}{}
{\noindent\textcolor{FuncColor}{$\triangleright$\ \ \texttt{KSKnapsack1({\mdseries\slshape K})\index{KSKnapsack1@\texttt{KSKnapsack1}}
\label{KSKnapsack1}
}\hfill{\scriptsize (function)}}\\


 Implementation of Algorithm 4.1. \mbox{\texttt{\mdseries\slshape K}} is a list, which elements are profits, weights, capacity. }

 

\subsection{\textcolor{Chapter }{KSKnapsack2}}
\logpage{[ 2, 1, 3 ]}\nobreak
\hyperdef{L}{X7D730B657AEFF5DB}{}
{\noindent\textcolor{FuncColor}{$\triangleright$\ \ \texttt{KSKnapsack2({\mdseries\slshape K})\index{KSKnapsack2@\texttt{KSKnapsack2}}
\label{KSKnapsack2}
}\hfill{\scriptsize (function)}}\\


 Implementation of Algorithm 4.3. \mbox{\texttt{\mdseries\slshape K}} is a list, which elements are profits, weights, capacity. }

 }

 
\section{\textcolor{Chapter }{Generating all cliques}}\label{allcliques}
\logpage{[ 2, 2, 0 ]}
\hyperdef{L}{X8514296C86C716C8}{}
{
  

\subsection{\textcolor{Chapter }{KSAllCliques}}
\logpage{[ 2, 2, 1 ]}\nobreak
\hyperdef{L}{X86D20A268630B591}{}
{\noindent\textcolor{FuncColor}{$\triangleright$\ \ \texttt{KSAllCliques({\mdseries\slshape graph})\index{KSAllCliques@\texttt{KSAllCliques}}
\label{KSAllCliques}
}\hfill{\scriptsize (function)}}\\


 Implementation of Algorithm 4.4. A graph $G$ is defined by the list \mbox{\texttt{\mdseries\slshape graph}}, which must be a list of subsets of $\{1,...,n\}$, for some integer $n$. The neighbors of vertex $i$ are the elements of \mbox{\texttt{\mdseries\slshape graph[i]}}. }

 }

 
\section{\textcolor{Chapter }{Exact cover}}\label{exactcover}
\logpage{[ 2, 3, 0 ]}
\hyperdef{L}{X83868AF47AD4244E}{}
{
  

\subsection{\textcolor{Chapter }{KSExactCover}}
\logpage{[ 2, 3, 1 ]}\nobreak
\hyperdef{L}{X86E5032879288555}{}
{\noindent\textcolor{FuncColor}{$\triangleright$\ \ \texttt{KSExactCover({\mdseries\slshape number, cover})\index{KSExactCover@\texttt{KSExactCover}}
\label{KSExactCover}
}\hfill{\scriptsize (function)}}\\


 Finds an subcollection of \mbox{\texttt{\mdseries\slshape cover}} (which is a set of subsets of $\{1,..,\mbox{\texttt{\mdseries\slshape number}}\}$) that is an exact cover of $\{1,..,\mbox{\texttt{\mdseries\slshape number}}\}$, if it exists. }

 

\subsection{\textcolor{Chapter }{KSRandomSubsetOfSubsets}}
\logpage{[ 2, 3, 2 ]}\nobreak
\hyperdef{L}{X79F75FD58281A839}{}
{\noindent\textcolor{FuncColor}{$\triangleright$\ \ \texttt{KSRandomSubsetOfSubsets({\mdseries\slshape n, delta})\index{KSRandomSubsetOfSubsets@\texttt{KSRandomSubsetOfSubsets}}
\label{KSRandomSubsetOfSubsets}
}\hfill{\scriptsize (function)}}\\


 Generates a random subset of the set of all subsets of $\{1..\mbox{\texttt{\mdseries\slshape n}}\}$, with density \mbox{\texttt{\mdseries\slshape delta}}. This can be used as an instance of the ExactCover problem. }

 }

 
\section{\textcolor{Chapter }{Bounding functions}}\label{bounding-functions}
\logpage{[ 2, 4, 0 ]}
\hyperdef{L}{X7E9BE74B7B20F799}{}
{
  

\subsection{\textcolor{Chapter }{KSSortForRationalKnapsack}}
\logpage{[ 2, 4, 1 ]}\nobreak
\hyperdef{L}{X7C63779C8478C20C}{}
{\noindent\textcolor{FuncColor}{$\triangleright$\ \ \texttt{KSSortForRationalKnapsack({\mdseries\slshape K})\index{KSSortForRationalKnapsack@\texttt{KSSortForRationalKnapsack}}
\label{KSSortForRationalKnapsack}
}\hfill{\scriptsize (function)}}\\


 Given an instance \mbox{\texttt{\mdseries\slshape K}} of the Knapsack Problem, where the two first components of \mbox{\texttt{\mdseries\slshape K}} represent profits and weights, this function returns a list, where the first
component is the same instance of the problem, but the profits and weights
have been sorted in non-increasing order of values of $\mbox{\texttt{\mdseries\slshape profits[i]}}/\mbox{\texttt{\mdseries\slshape weights[i]}}$. The second component is the permutation applied to the original problem. }

 

\subsection{\textcolor{Chapter }{KSRationalKnapsackSorted}}
\logpage{[ 2, 4, 2 ]}\nobreak
\hyperdef{L}{X87F7D9DD7B0E9DE4}{}
{\noindent\textcolor{FuncColor}{$\triangleright$\ \ \texttt{KSRationalKnapsackSorted({\mdseries\slshape K})\index{KSRationalKnapsackSorted@\texttt{KSRationalKnapsackSorted}}
\label{KSRationalKnapsackSorted}
}\hfill{\scriptsize (function)}}\\


 Solves the rational Knapsack problem for the instance \mbox{\texttt{\mdseries\slshape K}}. Profits and weights must be sorted in non-increasing order of values of $\mbox{\texttt{\mdseries\slshape profits[i]}}/\mbox{\texttt{\mdseries\slshape weights[i]}}$. }

 

\subsection{\textcolor{Chapter }{KSRationalKnapsack}}
\logpage{[ 2, 4, 3 ]}\nobreak
\hyperdef{L}{X81A085627DA480A4}{}
{\noindent\textcolor{FuncColor}{$\triangleright$\ \ \texttt{KSRationalKnapsack({\mdseries\slshape K})\index{KSRationalKnapsack@\texttt{KSRationalKnapsack}}
\label{KSRationalKnapsack}
}\hfill{\scriptsize (function)}}\\


 Solves the rational Knapsack problem for the instance \mbox{\texttt{\mdseries\slshape K}}. }

 

\subsection{\textcolor{Chapter }{KSKnapsack3}}
\logpage{[ 2, 4, 4 ]}\nobreak
\hyperdef{L}{X7A03786C86F71827}{}
{\noindent\textcolor{FuncColor}{$\triangleright$\ \ \texttt{KSKnapsack3({\mdseries\slshape K})\index{KSKnapsack3@\texttt{KSKnapsack3}}
\label{KSKnapsack3}
}\hfill{\scriptsize (function)}}\\


 Solves the Knapsack problem for the instace \mbox{\texttt{\mdseries\slshape K}}, using the function KSRationalKnapsack as bounding function. }

 

\subsection{\textcolor{Chapter }{KSRandomKnapsackInstance}}
\logpage{[ 2, 4, 5 ]}\nobreak
\hyperdef{L}{X7F7AF9917C7513E0}{}
{\noindent\textcolor{FuncColor}{$\triangleright$\ \ \texttt{KSRandomKnapsackInstance({\mdseries\slshape size, maximum{\textunderscore}weight})\index{KSRandomKnapsackInstance@\texttt{KSRandomKnapsackInstance}}
\label{KSRandomKnapsackInstance}
}\hfill{\scriptsize (function)}}\\


 Returns a random instance of a Knapsack problem, for \mbox{\texttt{\mdseries\slshape size}} objects. The maximum weight is \mbox{\texttt{\mdseries\slshape maximum{\textunderscore}weight}}. For each $i$, the profit $P[i]$ is $2*W[i]*\epsilon$, where $\epsilon$ is a random number between $0.9$ and $1.1$. }

 

\subsection{\textcolor{Chapter }{KSRandomTSPInstance}}
\logpage{[ 2, 4, 6 ]}\nobreak
\hyperdef{L}{X824023CC7A1DAEEE}{}
{\noindent\textcolor{FuncColor}{$\triangleright$\ \ \texttt{KSRandomTSPInstance({\mdseries\slshape n, Wmax})\index{KSRandomTSPInstance@\texttt{KSRandomTSPInstance}}
\label{KSRandomTSPInstance}
}\hfill{\scriptsize (function)}}\\


 Returns a random instance of the TSP problem, which is a symmetric \mbox{\texttt{\mdseries\slshape n}} by \mbox{\texttt{\mdseries\slshape n}} matrix, such that its $ij$ entry is the cost to travel from city $i$ to city $j$. The entries in the diagonal are made equal to $\infty$. Each cost is a random integer between 1 and \mbox{\texttt{\mdseries\slshape Wmax}}. }

 

\subsection{\textcolor{Chapter }{KSTSP1}}
\logpage{[ 2, 4, 7 ]}\nobreak
\hyperdef{L}{X8208350A7F23DC4F}{}
{\noindent\textcolor{FuncColor}{$\triangleright$\ \ \texttt{KSTSP1({\mdseries\slshape G})\index{KSTSP1@\texttt{KSTSP1}}
\label{KSTSP1}
}\hfill{\scriptsize (function)}}\\


 Solves the TSP problem, for the instance \mbox{\texttt{\mdseries\slshape G}}, traversing the whole tree space. }

 

\subsection{\textcolor{Chapter }{KSMinCostBound}}
\logpage{[ 2, 4, 8 ]}\nobreak
\hyperdef{L}{X845476007D8005D4}{}
{\noindent\textcolor{FuncColor}{$\triangleright$\ \ \texttt{KSMinCostBound({\mdseries\slshape V, G})\index{KSMinCostBound@\texttt{KSMinCostBound}}
\label{KSMinCostBound}
}\hfill{\scriptsize (function)}}\\


 A bounding function for the TSP problem. }

 

\subsection{\textcolor{Chapter }{KSReduce}}
\logpage{[ 2, 4, 9 ]}\nobreak
\hyperdef{L}{X8683492082A8C4B4}{}
{\noindent\textcolor{FuncColor}{$\triangleright$\ \ \texttt{KSReduce({\mdseries\slshape M})\index{KSReduce@\texttt{KSReduce}}
\label{KSReduce}
}\hfill{\scriptsize (function)}}\\


 Reduce function for matrices, which will be useful to implement a secound
bounding function for the TSP problem. }

 

\subsection{\textcolor{Chapter }{KSReduceBound}}
\logpage{[ 2, 4, 10 ]}\nobreak
\hyperdef{L}{X7C74326D7CA8BA24}{}
{\noindent\textcolor{FuncColor}{$\triangleright$\ \ \texttt{KSReduceBound({\mdseries\slshape V, M})\index{KSReduceBound@\texttt{KSReduceBound}}
\label{KSReduceBound}
}\hfill{\scriptsize (function)}}\\


 A second bounding function for the TSP problem. \mbox{\texttt{\mdseries\slshape V}} is a partial solution, and \mbox{\texttt{\mdseries\slshape M}} is the problem instance. This implements Algorithm 4.12. }

 

\subsection{\textcolor{Chapter }{KSTSP2}}
\logpage{[ 2, 4, 11 ]}\nobreak
\hyperdef{L}{X7B98A011874A9B65}{}
{\noindent\textcolor{FuncColor}{$\triangleright$\ \ \texttt{KSTSP2({\mdseries\slshape G, F})\index{KSTSP2@\texttt{KSTSP2}}
\label{KSTSP2}
}\hfill{\scriptsize (function)}}\\


 Solves the TSP problem for instance \mbox{\texttt{\mdseries\slshape G}}, using the bounding function \mbox{\texttt{\mdseries\slshape F}}. }

 

\subsection{\textcolor{Chapter }{KSMaxClique1}}
\logpage{[ 2, 4, 12 ]}\nobreak
\hyperdef{L}{X7E670BAD7D143608}{}
{\noindent\textcolor{FuncColor}{$\triangleright$\ \ \texttt{KSMaxClique1({\mdseries\slshape G})\index{KSMaxClique1@\texttt{KSMaxClique1}}
\label{KSMaxClique1}
}\hfill{\scriptsize (function)}}\\


 Adapts the function that lists the complete subgraphs of \mbox{\texttt{\mdseries\slshape G}}, to find the size of the largest clique of \mbox{\texttt{\mdseries\slshape G}}. This implements Algorithm 4.14. }

 

\subsection{\textcolor{Chapter }{KSMaxClique2}}
\logpage{[ 2, 4, 13 ]}\nobreak
\hyperdef{L}{X87F79EB67A6DDB27}{}
{\noindent\textcolor{FuncColor}{$\triangleright$\ \ \texttt{KSMaxClique2({\mdseries\slshape G, F})\index{KSMaxClique2@\texttt{KSMaxClique2}}
\label{KSMaxClique2}
}\hfill{\scriptsize (function)}}\\


 Finds the size of the maximum clique in the graph \mbox{\texttt{\mdseries\slshape G}}, using the bounding function \mbox{\texttt{\mdseries\slshape F}}. This implements Algorithm 4.19. }

 

\subsection{\textcolor{Chapter }{KSSizeBound}}
\logpage{[ 2, 4, 14 ]}\nobreak
\hyperdef{L}{X7A6A4127818EDC0E}{}
{\noindent\textcolor{FuncColor}{$\triangleright$\ \ \texttt{KSSizeBound({\mdseries\slshape XX, G, Cl})\index{KSSizeBound@\texttt{KSSizeBound}}
\label{KSSizeBound}
}\hfill{\scriptsize (function)}}\\


 A bounding function for the MaxClique problem. \mbox{\texttt{\mdseries\slshape XX}} is a complete subgraph of \mbox{\texttt{\mdseries\slshape G}}, and \mbox{\texttt{\mdseries\slshape Cl}} is the set of candidates to extend \mbox{\texttt{\mdseries\slshape XX}}. }

 

\subsection{\textcolor{Chapter }{KSGenerateRandomGraph}}
\logpage{[ 2, 4, 15 ]}\nobreak
\hyperdef{L}{X78E238A984F3911E}{}
{\noindent\textcolor{FuncColor}{$\triangleright$\ \ \texttt{KSGenerateRandomGraph({\mdseries\slshape n})\index{KSGenerateRandomGraph@\texttt{KSGenerateRandomGraph}}
\label{KSGenerateRandomGraph}
}\hfill{\scriptsize (function)}}\\


 Returns a list of edges of a random graph on \mbox{\texttt{\mdseries\slshape n}} vertices. This implements Algorithm 4.20. }

 

\subsection{\textcolor{Chapter }{KSEdgeListToAdjacencyList}}
\logpage{[ 2, 4, 16 ]}\nobreak
\hyperdef{L}{X7C0E333A7B200D4C}{}
{\noindent\textcolor{FuncColor}{$\triangleright$\ \ \texttt{KSEdgeListToAdjacencyList({\mdseries\slshape Ged, n})\index{KSEdgeListToAdjacencyList@\texttt{KSEdgeListToAdjacencyList}}
\label{KSEdgeListToAdjacencyList}
}\hfill{\scriptsize (function)}}\\


 Given the list of edges \mbox{\texttt{\mdseries\slshape Ged}} of a graph with \mbox{\texttt{\mdseries\slshape n}} vertices, returns the adjacency list of such graph. }

 

\subsection{\textcolor{Chapter }{KSGreedyColor}}
\logpage{[ 2, 4, 17 ]}\nobreak
\hyperdef{L}{X7C4BB3897802A465}{}
{\noindent\textcolor{FuncColor}{$\triangleright$\ \ \texttt{KSGreedyColor({\mdseries\slshape G})\index{KSGreedyColor@\texttt{KSGreedyColor}}
\label{KSGreedyColor}
}\hfill{\scriptsize (function)}}\\


 Colors the vertices of a graph \mbox{\texttt{\mdseries\slshape G}} using a greedy strategy. This implements Algorithm 4.16. }

 

\subsection{\textcolor{Chapter }{KSSamplingBound}}
\logpage{[ 2, 4, 18 ]}\nobreak
\hyperdef{L}{X83A5C32080EA9396}{}
{\noindent\textcolor{FuncColor}{$\triangleright$\ \ \texttt{KSSamplingBound({\mdseries\slshape XX, G, Cl})\index{KSSamplingBound@\texttt{KSSamplingBound}}
\label{KSSamplingBound}
}\hfill{\scriptsize (function)}}\\


 A bounding function for the MaxClique problem. \mbox{\texttt{\mdseries\slshape XX}} is a complete subgraph of \mbox{\texttt{\mdseries\slshape G}}, and \mbox{\texttt{\mdseries\slshape Cl}} is the set of candidates to extend \mbox{\texttt{\mdseries\slshape XX}}. This function uses a fixed greedy coloring of the graph \mbox{\texttt{\mdseries\slshape G}}. Implements Algorithm 4.17. }

 

\subsection{\textcolor{Chapter }{KSInducedSubgraph}}
\logpage{[ 2, 4, 19 ]}\nobreak
\hyperdef{L}{X800A04C2811D4374}{}
{\noindent\textcolor{FuncColor}{$\triangleright$\ \ \texttt{KSInducedSubgraph({\mdseries\slshape G, L})\index{KSInducedSubgraph@\texttt{KSInducedSubgraph}}
\label{KSInducedSubgraph}
}\hfill{\scriptsize (function)}}\\


 Returns the adjacency list of the subgraph of \mbox{\texttt{\mdseries\slshape G}} induced by the vertices in \mbox{\texttt{\mdseries\slshape L}}. }

 

\subsection{\textcolor{Chapter }{KSGreedyBound}}
\logpage{[ 2, 4, 20 ]}\nobreak
\hyperdef{L}{X807E30DC7CDAD9D3}{}
{\noindent\textcolor{FuncColor}{$\triangleright$\ \ \texttt{KSGreedyBound({\mdseries\slshape XX, G, Cl})\index{KSGreedyBound@\texttt{KSGreedyBound}}
\label{KSGreedyBound}
}\hfill{\scriptsize (function)}}\\


 A bounding function for the MaxClique problem. \mbox{\texttt{\mdseries\slshape XX}} is a complete subgraph of \mbox{\texttt{\mdseries\slshape G}}, and \mbox{\texttt{\mdseries\slshape Cl}} is the set of candidates to extend \mbox{\texttt{\mdseries\slshape XX}}. This uses a greedy coloring of the subgraph of \mbox{\texttt{\mdseries\slshape G}} induced by \mbox{\texttt{\mdseries\slshape L}}. }

 

\subsection{\textcolor{Chapter }{KSGenerateRandomGraph2}}
\logpage{[ 2, 4, 21 ]}\nobreak
\hyperdef{L}{X7A8E9FCD8238638C}{}
{\noindent\textcolor{FuncColor}{$\triangleright$\ \ \texttt{KSGenerateRandomGraph2({\mdseries\slshape n, delta})\index{KSGenerateRandomGraph2@\texttt{KSGenerateRandomGraph2}}
\label{KSGenerateRandomGraph2}
}\hfill{\scriptsize (function)}}\\


 Returns the list of edges of a random graph on \mbox{\texttt{\mdseries\slshape n}} vertices with edge density \mbox{\texttt{\mdseries\slshape delta}}. }

 

\subsection{\textcolor{Chapter }{KSTSP3}}
\logpage{[ 2, 4, 22 ]}\nobreak
\hyperdef{L}{X7CE8D3187BFF565F}{}
{\noindent\textcolor{FuncColor}{$\triangleright$\ \ \texttt{KSTSP3({\mdseries\slshape G, F})\index{KSTSP3@\texttt{KSTSP3}}
\label{KSTSP3}
}\hfill{\scriptsize (function)}}\\


 Solves the TSP problem for instance \mbox{\texttt{\mdseries\slshape G}}, using bounding function \mbox{\texttt{\mdseries\slshape F}}, applying the branch and bound technique. }

 }

 
\section{\textcolor{Chapter }{Exercises}}\label{exercises-backtracking}
\logpage{[ 2, 5, 0 ]}
\hyperdef{L}{X823A8E467E2D876D}{}
{
  

\subsection{\textcolor{Chapter }{KSQueens}}
\logpage{[ 2, 5, 1 ]}\nobreak
\hyperdef{L}{X86BA078279D66D0C}{}
{\noindent\textcolor{FuncColor}{$\triangleright$\ \ \texttt{KSQueens({\mdseries\slshape size})\index{KSQueens@\texttt{KSQueens}}
\label{KSQueens}
}\hfill{\scriptsize (function)}}\\


 Solves the $n$ queens problem for a $\mbox{\texttt{\mdseries\slshape size}}\times\mbox{\texttt{\mdseries\slshape size}}$ board. (Exercise 4.1.(a)) 
\begin{Verbatim}[commandchars=!@|,fontsize=\small,frame=single,label=Example]
  !gapprompt@gap>| !gapinput@KSQueens(4);|
  [ 2, 4, 1, 3 ]
  [ 3, 1, 4, 2 ]
\end{Verbatim}
 }

 

\subsection{\textcolor{Chapter }{KSWalks}}
\logpage{[ 2, 5, 2 ]}\nobreak
\hyperdef{L}{X869469F9844C627C}{}
{\noindent\textcolor{FuncColor}{$\triangleright$\ \ \texttt{KSWalks({\mdseries\slshape number})\index{KSWalks@\texttt{KSWalks}}
\label{KSWalks}
}\hfill{\scriptsize (function)}}\\


 Finds all non-overlapping walks in the plane of length \mbox{\texttt{\mdseries\slshape number}}. (Exercise 4.1.(b)) }

 }

 }

 
\chapter{\textcolor{Chapter }{Heuristic Search}}\label{heuristic}
\logpage{[ 3, 0, 0 ]}
\hyperdef{L}{X7A00AAD585C388D4}{}
{
  
\section{\textcolor{Chapter }{Uniform graph partition}}\label{uniform}
\logpage{[ 3, 1, 0 ]}
\hyperdef{L}{X7FF662447A34596B}{}
{
  

\subsection{\textcolor{Chapter }{KSRandomkSubset}}
\logpage{[ 3, 1, 1 ]}\nobreak
\hyperdef{L}{X7FDB1D757B321B26}{}
{\noindent\textcolor{FuncColor}{$\triangleright$\ \ \texttt{KSRandomkSubset({\mdseries\slshape k, n})\index{KSRandomkSubset@\texttt{KSRandomkSubset}}
\label{KSRandomkSubset}
}\hfill{\scriptsize (function)}}\\


 Returns a randomly chosen \mbox{\texttt{\mdseries\slshape k}}-subset of the set of integers from 1 to \mbox{\texttt{\mdseries\slshape n}}. }

 

\subsection{\textcolor{Chapter }{KSSelectPartition}}
\logpage{[ 3, 1, 2 ]}\nobreak
\hyperdef{L}{X79D69A0584EEE941}{}
{\noindent\textcolor{FuncColor}{$\triangleright$\ \ \texttt{KSSelectPartition({\mdseries\slshape n})\index{KSSelectPartition@\texttt{KSSelectPartition}}
\label{KSSelectPartition}
}\hfill{\scriptsize (function)}}\\


 Returns a random partition of the set $\{1,2,\ldots,2n\}$ into two subsets of size \mbox{\texttt{\mdseries\slshape n}} each. (Algorithm 5.7) }

 

\subsection{\textcolor{Chapter }{KSCost}}
\logpage{[ 3, 1, 3 ]}\nobreak
\hyperdef{L}{X7D5EA8347E9AE782}{}
{\noindent\textcolor{FuncColor}{$\triangleright$\ \ \texttt{KSCost({\mdseries\slshape G, P})\index{KSCost@\texttt{KSCost}}
\label{KSCost}
}\hfill{\scriptsize (function)}}\\


 Returns the cost of the partition \mbox{\texttt{\mdseries\slshape P}} of the vertices of the weighted graph \mbox{\texttt{\mdseries\slshape G}}. }

 

\subsection{\textcolor{Chapter }{KSGain}}
\logpage{[ 3, 1, 4 ]}\nobreak
\hyperdef{L}{X81D593AB78665091}{}
{\noindent\textcolor{FuncColor}{$\triangleright$\ \ \texttt{KSGain({\mdseries\slshape G, P, u, v})\index{KSGain@\texttt{KSGain}}
\label{KSGain}
}\hfill{\scriptsize (function)}}\\


 \mbox{\texttt{\mdseries\slshape P}} is a partition in equal parts of the vertices of \mbox{\texttt{\mdseries\slshape G}}. This function calculates the change in the value of the cost function when
interchanging the vertex \mbox{\texttt{\mdseries\slshape u}} from the first set in the partition \mbox{\texttt{\mdseries\slshape P}} with the vertex \mbox{\texttt{\mdseries\slshape v}} which is in the second set of the partition. }

 

\subsection{\textcolor{Chapter }{KSRandomCostMatrix}}
\logpage{[ 3, 1, 5 ]}\nobreak
\hyperdef{L}{X7F66F41C836329D7}{}
{\noindent\textcolor{FuncColor}{$\triangleright$\ \ \texttt{KSRandomCostMatrix({\mdseries\slshape n, Wmax})\index{KSRandomCostMatrix@\texttt{KSRandomCostMatrix}}
\label{KSRandomCostMatrix}
}\hfill{\scriptsize (function)}}\\


 Returns a symmetric \mbox{\texttt{\mdseries\slshape n}} by \mbox{\texttt{\mdseries\slshape n}} matrix, such that its entries are random integers from 0 to \mbox{\texttt{\mdseries\slshape Wmax}}, and with zeros in the main diagonal. }

 

\subsection{\textcolor{Chapter }{KSAscend}}
\logpage{[ 3, 1, 6 ]}\nobreak
\hyperdef{L}{X8418E55085D15E1F}{}
{\noindent\textcolor{FuncColor}{$\triangleright$\ \ \texttt{KSAscend({\mdseries\slshape G, P})\index{KSAscend@\texttt{KSAscend}}
\label{KSAscend}
}\hfill{\scriptsize (function)}}\\


 Given a partition \mbox{\texttt{\mdseries\slshape P}} of the vertices of the weighted graph \mbox{\texttt{\mdseries\slshape G}}, it returns a partition \mbox{\texttt{\mdseries\slshape Q}} with less cost than \mbox{\texttt{\mdseries\slshape P}}, by exchanging one vertex of the partition, if such partition exists.
Otherwise, returns the same partition \mbox{\texttt{\mdseries\slshape P}}. }

 }

 
\section{\textcolor{Chapter }{Steiner systems}}\label{steiner}
\logpage{[ 3, 2, 0 ]}
\hyperdef{L}{X79DB42D47FB0C42A}{}
{
  

\subsection{\textcolor{Chapter }{KSConstructBlocks}}
\logpage{[ 3, 2, 1 ]}\nobreak
\hyperdef{L}{X7F97E5CB831AA666}{}
{\noindent\textcolor{FuncColor}{$\triangleright$\ \ \texttt{KSConstructBlocks({\mdseries\slshape v, other})\index{KSConstructBlocks@\texttt{KSConstructBlocks}}
\label{KSConstructBlocks}
}\hfill{\scriptsize (function)}}\\


 Constructs a list of blocks of length \mbox{\texttt{\mdseries\slshape v}} from the list of lists \mbox{\texttt{\mdseries\slshape other}}. (Algorithm 5.12) }

 

\subsection{\textcolor{Chapter }{KSRevisedStinsonAlgorithm}}
\logpage{[ 3, 2, 2 ]}\nobreak
\hyperdef{L}{X84C3665F829A89E6}{}
{\noindent\textcolor{FuncColor}{$\triangleright$\ \ \texttt{KSRevisedStinsonAlgorithm({\mdseries\slshape v})\index{KSRevisedStinsonAlgorithm@\texttt{KSRevisedStinsonAlgorithm}}
\label{KSRevisedStinsonAlgorithm}
}\hfill{\scriptsize (function)}}\\


 Constructs a Steiner triple system with \mbox{\texttt{\mdseries\slshape v}} points, using a hill-climbing algorithm. Implements Algorithm 5.19. }

 }

 
\section{\textcolor{Chapter }{The knapsack problem}}\label{The knapsack problem}
\logpage{[ 3, 3, 0 ]}
\hyperdef{L}{X8165BE5D7D465211}{}
{
  

\subsection{\textcolor{Chapter }{KSKnapsackSimulatedAnnealing}}
\logpage{[ 3, 3, 1 ]}\nobreak
\hyperdef{L}{X7D54A90C7F631FF5}{}
{\noindent\textcolor{FuncColor}{$\triangleright$\ \ \texttt{KSKnapsackSimulatedAnnealing({\mdseries\slshape K, cmax, T0, alpha})\index{KSKnapsackSimulatedAnnealing@\texttt{KSKnapsackSimulatedAnnealing}}
\label{KSKnapsackSimulatedAnnealing}
}\hfill{\scriptsize (function)}}\\


 Implements Algorithm 5.20. \mbox{\texttt{\mdseries\slshape K}} is the instance of the Knapsack problem to solve. \mbox{\texttt{\mdseries\slshape cmax}} is the number of iterations to be done. \mbox{\texttt{\mdseries\slshape T0}} is the initial "temperature" and \mbox{\texttt{\mdseries\slshape alpha}} is the parameter of the "cooling schedule". }

 

\subsection{\textcolor{Chapter }{KSRandomFeasibleSolutionKnapsack}}
\logpage{[ 3, 3, 2 ]}\nobreak
\hyperdef{L}{X7B069B0D7CAC98E0}{}
{\noindent\textcolor{FuncColor}{$\triangleright$\ \ \texttt{KSRandomFeasibleSolutionKnapsack({\mdseries\slshape K})\index{KSRandomFeasibleSolutionKnapsack@\texttt{KSRandomFeasibleSolutionKnapsack}}
\label{KSRandomFeasibleSolutionKnapsack}
}\hfill{\scriptsize (function)}}\\


 Returns a randomly chosen feasible solution to the Knapsack problem instance \mbox{\texttt{\mdseries\slshape K}}. }

 

\subsection{\textcolor{Chapter }{KSKnapsackTabuSearch}}
\logpage{[ 3, 3, 3 ]}\nobreak
\hyperdef{L}{X7A1AD5D47A2522CA}{}
{\noindent\textcolor{FuncColor}{$\triangleright$\ \ \texttt{KSKnapsackTabuSearch({\mdseries\slshape K, cmax, L})\index{KSKnapsackTabuSearch@\texttt{KSKnapsackTabuSearch}}
\label{KSKnapsackTabuSearch}
}\hfill{\scriptsize (function)}}\\


 Searches for an optimal solution to the Knapsack problem instance \mbox{\texttt{\mdseries\slshape K}} using a tabu search list. \mbox{\texttt{\mdseries\slshape cmax}} is the maximum number of iterations, and \mbox{\texttt{\mdseries\slshape L}} the length of iterations a tabu search should be kept. }

 }

 
\section{\textcolor{Chapter }{Heuristics for the TSP}}\label{Heuristics for the TSP}
\logpage{[ 3, 4, 0 ]}
\hyperdef{L}{X878B948E7DF6D33B}{}
{
  

\subsection{\textcolor{Chapter }{KSGainTSP}}
\logpage{[ 3, 4, 1 ]}\nobreak
\hyperdef{L}{X7F3649047B6F7BCF}{}
{\noindent\textcolor{FuncColor}{$\triangleright$\ \ \texttt{KSGainTSP({\mdseries\slshape XX, i, j, M})\index{KSGainTSP@\texttt{KSGainTSP}}
\label{KSGainTSP}
}\hfill{\scriptsize (function)}}\\


 Gain function for the Traveling Salesman Problem. }

 

\subsection{\textcolor{Chapter }{KSSteepestAscentTwoOpt}}
\logpage{[ 3, 4, 2 ]}\nobreak
\hyperdef{L}{X7C8E3F5B7E90B1D1}{}
{\noindent\textcolor{FuncColor}{$\triangleright$\ \ \texttt{KSSteepestAscentTwoOpt({\mdseries\slshape XX, M})\index{KSSteepestAscentTwoOpt@\texttt{KSSteepestAscentTwoOpt}}
\label{KSSteepestAscentTwoOpt}
}\hfill{\scriptsize (function)}}\\


 Given an instance of the TSP problem \mbox{\texttt{\mdseries\slshape M}}, and an initial permutation $XX$, applies steepest ascent heuristic. }

 

\subsection{\textcolor{Chapter }{KSSelect}}
\logpage{[ 3, 4, 3 ]}\nobreak
\hyperdef{L}{X8104AEAC7A8F9599}{}
{\noindent\textcolor{FuncColor}{$\triangleright$\ \ \texttt{KSSelect({\mdseries\slshape popsize, M})\index{KSSelect@\texttt{KSSelect}}
\label{KSSelect}
}\hfill{\scriptsize (function)}}\\


 Returns a population of size \mbox{\texttt{\mdseries\slshape popsize}} for the TSP problem \mbox{\texttt{\mdseries\slshape M}}. }

 

\subsection{\textcolor{Chapter }{KSPartiallyMatchedCrossover}}
\logpage{[ 3, 4, 4 ]}\nobreak
\hyperdef{L}{X7A6F4A4E83614C33}{}
{\noindent\textcolor{FuncColor}{$\triangleright$\ \ \texttt{KSPartiallyMatchedCrossover({\mdseries\slshape n, alpha, beta, j, k})\index{KSPartiallyMatchedCrossover@\texttt{KSPartiallyMatchedCrossover}}
\label{KSPartiallyMatchedCrossover}
}\hfill{\scriptsize (function)}}\\


 One way to obtain two new permutations from permutations \mbox{\texttt{\mdseries\slshape alpha}}, \mbox{\texttt{\mdseries\slshape beta}}. }

 }

 }

 \def\indexname{Index\logpage{[ "Ind", 0, 0 ]}
\hyperdef{L}{X83A0356F839C696F}{}
}

\cleardoublepage
\phantomsection
\addcontentsline{toc}{chapter}{Index}


\printindex

\newpage
\immediate\write\pagenrlog{["End"], \arabic{page}];}
\immediate\closeout\pagenrlog
\end{document}
